\documentclass{article}

\usepackage{amsmath}
\usepackage[top=0.5in, bottom=1in]{geometry}

\setlength\parindent{0cm}
\setlength\parskip{1.5ex}

\def\p#1{ \left( #1 \right)}
\def\norm#1{ \left\lVert #1 \right\rVert}

\begin{document}
This file documents the symbols, formulae and equations used in this module.

\section{Symbols}

\begin{itemize}
	\item $a$: length of semimajor axis, mean of periapsis and apoapsis
	\item $b$: length of semiminor axis
	\item $E$: Eccentric anomaly
	\item $e$: Eccentricity. In this file, if there are any references to Euler's number, it is denoted as $\operatorname{exp}(1)$ etc..
	\item $h$: Angular momentum
	\item $M$: Mean anomaly
	\item $n$: Average sweep, the average angular velocity per unit time
	\item $\nu$: True anomaly, angle [periapsis]-[star]-[body]
	\item $p$: Semi-latus rectum, length from star to orbit parallel to minor axis
	\item $r$: Distance from star
	\item $\mathbf r$: Position relative to star
	\item $\theta$: The argument of a body in the standard polar coordinate system
	\item $u$: Argument of latitude
	\item $v$: Magnitude of velocity
	\item $\mathbf v$: Velocity
	\item $\omega$: Longitude of periapsis, or equivalently the argument of periapsis, assuming longitutde of ascending node is zero in 2D
	\item $\epsilon$: Specific orbital energy, sum of potential and kinetic energy
\end{itemize}

\section{Equations}
\subsection{Orbital states to orbital elements}
In 2D, for a star with fixed position (the origin) and mass, orbits have 4 degrees of freedom.

Defining the ellipse requires 3 degrees of freedom:

\begin{itemize}
	\item Shape: fixed with $e$
	\item Size: fixed with $a$
	\item Rotation: fixed with $\omega$
\end{itemize}

For efficient computation over time, the position on the ellipse is represented by $M$.

The orbital states ($\mathbf r, \mathbf v$), with 4 degrees of freedom,
are bijective to the orbital elements.
Based on cart2kep2002.pdf
\footnote{https://web.archive.org/web/20160418175843/https://ccar.colorado.edu/asen5070/handouts/cart2kep2002.pdf},
the following procedures convert Cartesian orbital states to the orbital elements in 2D,
based on the external parameter $\mu$:

\begin{align*}
	h &= r_x v_y - r_y v_x \\
	\epsilon &= \frac12 \norm{\mathbf v}^2 - \frac\mu{\norm{\mathbf r}} \\
	a &= \frac{-\mu}{2\epsilon} \\
	e &= \sqrt{1 - \frac{h^2}{a\mu}} \\
	u &= \\
	p &= a (1 - e^2) \\
	\nu &= \arccos \frac{p - \norm{\mathbf r}}{\norm{\mathbf r}e} \\
	\nu &= \arctan \frac{\sqrt{\frac p\mu} (\mathbf r \cdot \mathbf v)}{p - \norm{\mathbf r}} \\
	\omega &= u - \nu \\
	E &= 2 \arctan \p{\sqrt{\frac{1-e}{1+e}} \tan \frac\nu2} \\
	M &= E - e \sin E
\end{align*}

This gives $(e, a, \omega, M)$ as the orbital elements.

\subsection{Change of mean anomaly over time}
Given mean anomaly $M$ at time $t$,
the new mean anomaly $M'$ at time $t'$ can be found by
$$ M' = M + n (t' - t), $$
where the average sweep $n$ can be found with
$$ n = \sqrt{\frac\mu{a^3}}. $$

\subsection{Radius}

\subsubsection{Testing if orbit is above/below radius}
For a frequently queried radius $r_q$,
precompute the (smaller) mean anomaly $M_q$ such that
$ r(M_q) = r(2\pi - M_q) = r_q$.
Then $r(t) > r_q \iff M_q < M(t) < 2\pi - M_q \pmod{2\pi}$.

For a specific $r_q$,
solve $\cos E_q = \frac1e \p{1 - \frac ra}$ for $E_q$.
Then calculate $M_q = E_q - e \sin E_q$.

Verification: $\cos E_q = \cos(2 \pi - E_q)$ are the two eccentric anomalies
where the star moves out of/into the queried radius.
$$ 2\pi - E_q - e \sin(2\pi - E_q) = 2\pi - \p{E_q - e \sin E_q} = 2\pi - M_q $$

\subsection{Bearing}
Bearing is the eviov name for argument in the polar coordinate system.

The bearing of the periapsis is $\omega$.
The bearing at specific true anomaly $\nu$ is $\omega + \nu$.

\subsubsection{Approximating the bearing}
This is done by direct approximation of true anomaly by Fourier expansion:

$$ \theta = \omega + \nu \approx \omega + M + \sum_{i=1}^\infty TODO e^i \sin(iM) $$

\subsubsection{Testing if orbit is between bearing}
For a frequently queried bearing range $\p{\theta_1, \theta_2}$,
precompute the mean anomalies $M_1, M_2$ such that
$\theta(M_1) = \theta_1$ and $\theta(M_2) = \theta_2$.
Then $\theta(t) \in \p{\theta_1, \theta_2} \iff M(t) \in \p{M_1, M_2}$.

Compute $$ E_q = 2 \arctan\p{ \sqrt{\frac{1-e}{1+e}} \tan\frac{\theta_q - \omega}2 } $$
for $q = 1,2$.
Then $M_q = E_q - e \sin E_q$.

\end{document}
